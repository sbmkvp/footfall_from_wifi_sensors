The aims of the main study are,
test the validity of the signal strength algorithm in different 
micro site conditions.
Test that the sequence number algorithm works in real world
for different locations and different times.
check if the thresholds are consistent.
Test if the calibration works over intervals
Finally conclude if we can estimate footfall confidently
with just probe requests.

five locations were selected across central london which
had different types of configurations and specific problems
configuratons are shown in fig. map is shown in figure.
\begin{enumerate}
	\item Phone Shop Camden - has phones and bus stops.
	\item Restaurant TCR - has seating area on either side.
	\item Holborn Information Kiosk - High volumne station entry
	\item Restaurant Russell Square - seating on one side and side walk on other
	\item Shop Charring Cross - sidewalk on one side and phone shop next door
\end{enumerate}
Installations were carried out over the time period from xxxx to xxxx.
the data collection happened from xxxx to xxxx. Manual counting was carried out
with high precision on dates xxxx and aggregated five minutes on xxxx.
The difference in methods could lead to some inaccuracies in data.
The overall schedule is shown in half page graphic.

The overall statistics of data collected. How many probe requests.
Comparision of location in terms of volume, patterns in daily footfall etc.
The comparision between global and local.
comparision between different types of vendors.
Specifics on top 5 manufacturers.

We do a daily analysis of distribution of signal strengths.
The thresholds are shown in the table.
The average is xxxx and standard deviation is xxx.
we notice that the variation is lot. There is a definite
change with the micro site locations.

We see how the signal strength filtering affects the counts.
compared to manual counts we look at the average mean errors (daily)
per 5 minutes. The counts go as follows.
Shown as a red line in the Figure.

We can conclude that even though it has variations, this is
a good method to reduce the overall level of error.

This is also done for different locations hourly for all the data we had.
we compare it to the manual counts and see that the average mean error
has been reduced/increased. The finger print works well for all the locations.
It also works over a period of time and gives us a comparable and close 
footfall count to the manual count. The thresholds found in the pilot study
works as well.

Finally we normalise the sensor counts to match the manual count using
a fraction/ adjustment factor calculated from the know manual counts.
we have three sets of counts. We check if the adjustment factor holds the
same in all three counts across locations. It does with a variations from
xxx to xxx. The results are shown in the table. 

we see that the signal strength filtering works and reduces error by
xxxxx. there is variation by locations.
we see that sequence number algorithm works as well. The threshold stays
constant well and works well across locations. The calibration also works
and ajustment factor stays consistent short term. This needs more work 
long term.
