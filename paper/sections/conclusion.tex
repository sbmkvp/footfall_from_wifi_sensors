It is important to note that the filtering process was done based soley on the information present in the probe requests and their temporal distribution.
This ensured that although the mobile devices were uniquely identified, there was no further personal data generated by linking the probe requests to the users of the mobile devices.
This method essentially gave us a way to estimate the footfall in real-time without identifying or tracking the mobile devices themselves.

This Wi-Fi based footfall counting methodology offers a large number of applications and benefits for real time spatial analysis.
Since Wi-Fi based sensors are inexpensive and the data model is scalable, it is possible to use this methodology for a large network of sensors to gather granular data on pedestrian footfall.
Projects such as SmartStreetSensors \citep{sss2016}, may utilise this methodology to overcome the challenges introduced by the implementation of MAC address randomisation.
Such precise and granular data also enables us to confidently model the pedestrian flow in urban road networks, and will be an indispensable tool in the smart city framework.
It can also be used to understand and classify geographical areas based on the spatio-temporal distribution of the volume of activity in them.
