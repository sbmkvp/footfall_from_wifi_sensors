There have been numerous attempts at using Wi-Fi to measure the volume and movement of people in the built environment for various applications \citep{zarim2006,sap2015,reki2007}.
Though most research obtains feasible and favorable results, in recent years, one of the major challenges faced in such attempts has been the MAC address randomisation process.
This process aims to protect the users’ privacy by anonymising the only globally identifiable portion of the probe requests, which results in a set of probe requests generated by the same device with different random MAC addresses \citep{green2008}.
There have been various successful attempts by researchers to breaking this randomisation process in order to extract real MAC addresses, \citep{martin2017} but this usually results in serious risk of infringement of the privacy of the users of the mobile devices.
There is a clear gap in the research for exploring methodologies which enable us to estimate the number of unique mobile devices from a set of anonymised probe requests, without the need to reveal their original MAC addresses.
