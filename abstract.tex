% ==============================================================================
% DOCUMENT HEADER
% ==============================================================================

\documentclass[11t, a4paper, twocolumn]{article} 
\input{structure.tex}

\title{Estimating real-time highstreet footfall from the Wi-Fi probe requests}
\author{
	\authorstyle{
		Balamurugan Soundararaj\textsuperscript{1}, 
		James Cheshire\textsuperscript{1} and 
		Paul Longley\textsuperscript{1}}
	\newline\newline
	\textsuperscript{1}\institution{
		Department of Geography, 
		University College London, 
		United Kingdom}
}
\date{\today}

% ==============================================================================
% DOCUMENT BODY
% ==============================================================================

\begin{document}

	% ---------------------------------------------------------------------------
	% We talk about wifi and probe requests. Possibility of 
	% counting people without revealing identity. There is uncertainity. We look
	% at resolving these. We have a setup to collect data from sensor and manual
	% We propose methodology to clean sensor count. compare with manual to see
	% it corresponds properly.
	% ---------------------------------------------------------------------------

	\maketitle
	\thispagestyle{firstpage}

	\textbf{Abstract}

	Wi-Fi has become an ubiquitous technology used to provide internet access in
	public and private spaces to people's mobile devices such as smartphones, 
	tablets and laptops. Any point in a dense built environment such as cities 
	has multiple Wi-Fi networks. Anticipating and remembering these networks
	mobile devices, to be able to switch seamlessly between them,
	broadcast a special type of management packets known as probe requests.
	This is a low level package specified in IEEE 802.1 specification
	which relays information about the source mobile device to
	any Access Points (AP) listening to them.
	This often the first steo in establishing a handshake between these devices,
	These probe requests are constinuous, passive and wireless stream of
	infromation available at any urban location which can act as a proxy to
	understanding the number of people present in that area in real-time.
	In this paper, utilising a set of probe requests collected
	at a highstreet location, along with manually collected data, we demonstrate
	that highstreet footfall can be estimated with reasonable accuracy without
	infringing on people's privacy.

	Use of wireless technology to measure movement of people in built
	environment has been done previously. It is either done at mobile
	level or network level.
	At network level it can be done with recording mobile devices coming in
	and out of network or measuing mobile devices recording aps.
	These can be done with various technologies such as gps, mobile network
	where some of them are more accurate than others.
	In this Wi-Fi provides us a good middle ground - interms of cost,
	scalability,  etc. The major disadvantage is that the unique identifier
	in wifi based counting is the MAC address which is global and can 
	cause privacy risk. This can be solved by hashing the personal data
	collected. Recently mobile devices have also started randomising their
	mac to avoid detection and tracking.
	Though this can be over come by using various techniques, cite papers,
	these methods are usually infringing on privacy.
	We need a method to count people using this data without tracking them
	which is the major challange in which we address here.

	The data collection is done through tshark, data collected were - mac,
	vendor, length, ssid, tags, signal, duration. - manual count was done in
	parallel through mobile phone app. The survey was carried out at oxford st.
	from 12:30 to 13:00. The total numbers. how many people. How many probes?
	how many unique macs? what are the uncertainities in the 

	
	internet access to mobile devices such as smartphones,tablets and laptops.
	In addition to internet these devices use these networks as a way to quick 
	geo location without waiting for GPS.
	To produce a list of WiFi networks available, mobile devices constantly 
	broadcast signals called probe requests.
	These signals have various information regarding the device and its 
	capabilities which are identifiable such as MAC addresses, non identifiable 
	such as signal strength and partly identifiable such as randomised mac 
	addresses.
	There is a significant use in uniquely identify the number of mobile devices
	at a specific location wihout actually affecting the privacy of the mobile
	devices.
	Such enumeration can be done by looking at identifiable information after 
	anonymising them whenever they are available and from the patterns in the 
	other information when they are not.
	We collect data on probe requests sent at set of locations in 
	different times and device a methodology for enumerating number of people
	at these locations from the data.
	We compare the enumerated counts with the counts collected manually at
	these locations to find that we are able to estimate real life with a 
	confidence of XX\%.
	
	\citep{vanhoef2016}
	\citep{matte2016}
	\citep{martin2017}
	\citep{vo2016}
	\citep{konto2017}

	\printbibliography[title={References}]

\end{document}
